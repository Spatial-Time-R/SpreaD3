\documentclass[english]{paper}


%%%%%%%%%%%%%%%%
%---PACKAGES---%
%%%%%%%%%%%%%%%%
\usepackage[T1]{fontenc}
\usepackage[latin9]{inputenc}
\usepackage{geometry}
\geometry{verbose,tmargin=3cm,bmargin=3cm,lmargin=2cm,rmargin=2cm,headheight=2cm,headsep=2cm,footskip=2cm}
\usepackage{float}
\usepackage{graphicx}
\usepackage{multicol}
\usepackage{hyperref}
\usepackage{babel}


%%%%%%%%%%%%%%%%%%%%%%%%%%%%%%%%%%%%%
%---USER SPECIFIED LaTeX COMMANDS---%
%%%%%%%%%%%%%%%%%%%%%%%%%%%%%%%%%%%%%

\makeatletter

\newenvironment{lyxcode}
{\par\begin{list}{}{
\setlength{\rightmargin}{\leftmargin}
\setlength{\listparindent}{0pt}% needed for AMS classes
\raggedright
\setlength{\itemsep}{0pt}
\setlength{\parsep}{0pt}
\normalfont\ttfamily}%
 \item[]}
{\end{list}}

\graphicspath{{figures/}}

% or sth else
\def \spreadname {SpreaD3}

\makeatother

%%%%%%%%%%%%%%%%
%---DOCUMENT---%
%%%%%%%%%%%%%%%%

\begin{document}

\title{{\spreadname}: Spatial Phylogenetic Reconstruction of Evolutionary Dynamics 3}
\maketitle

\begin{flushleft}
\textbf{Authors}
\par\end{flushleft}

\noindent
Filip Bielejec (\url{filip.bielejec(sorry_spybots)rega.kuleuven.be}) \\
Philippe Lemey (\url{philippe.lemey(sorry_spybots)rega.kuleuven.be}) \\
Andrew Rambaut (\url{a.rambaut(sorry_spybots)ed.ac.uk}) \\
Marc Suchard (\url{msuchard(sorry_spybots)ucla.edu})\\

%---TOC---%
\pagebreak{}
\tableofcontents{}
\pagebreak{}




% 1. Intro
% - Spread is 2-step analysis now 
% - we use JSON as a medium for the pipeline
% - a word of explanation what is JSON notation, how it can be read by
%   any O-O language (python, JS)
% - D3 (what it is etc), what is GEOJSON, how we use it as canvas for our
%   vis, where to get it from, maybe a couple of tips on manipulating
%   GeoJSON files (there are tools to merge, split etc).
% - software requirements 

% - still some support for KML and GE
% 
% 2. Examples of usage. Whese should show not just how but also why, what
% can we learn by visualising these processes.
% 
% 
% 2.1 MCC tree with discrete traits.
%  - Here we could use the EBOV as a case for study. All the steps Parse
% -> render -> manipulate the visualisation, screenshots etc. Place live links to data (hosted at github)   
% 
% 2.2 BF calculation and visualisation
%  - there seems to be a lot of confusion among folks as to what this
% actually does. I've seen them do calculations on binary coefficients
% other than indicators, or dismiss it thinking we do HME to get the BF
% we could use this to mitigate some of that confusion, so write what
% (and how) do we actually calculate these. I suppose a table in a text
% file is of main use but we also show the (graph) visualisation. H5N1 as
% an example?
% 
% - in location coordinates editor there's now 'generate button that will uniformly spread locations over a circle. Show an example of how this can be used to generate a vis (doesn't need a map layer, so no geojson needed). Perhaps more exciting than just a mere graph of BF values
%
% 2.3 MCC tree with Continuous traits
%  - languages example would look really nice here overlayed on a whole
% world map
% 
% 2.4 time-slicing 
% - West Nile Virus as an example?
% - show how one can use MCC tree to generate the slices or define
% his/her own slice heights
% - merge continuous tree with JSON file resulting form time slicing a
% posterior distribution for a joint visualisation.
% 
% 
% Tips & Tricks
%  - we should show the antigenic coordinates example here, where the
% coordinates are other than lat/long and there's no underlying map. I
% also have the multiple HPD's example, on which we can sho whow the
% merging works. W parse 3 JSONs for all HPD levels, merg ethem and
% jointly visualise, coloring polygons by HPD levels.
% - antigenic coord (or any other trait for taht matter) coul dbe plotted as a function of time (height attribute)
% - nodes/ branches of a given tree can be anotated in Figtree, these
%   values can then be used as a basis for mapping
% 
% - Some examples of KML rendering, using previously generated JSON files,
%   just to show that we still support it.




\bibliography{tutorefs} 

\end{document}
